\section{Виртуальная память}

\subsection{Виртуальная память: рапределение памяти страницами по запросам, схема с гиперстраницами, обоснование использования данной схемы. Управление памятью страницами по запросам в архитетуре x86 – расширенное преобразование (PAE) – схема преобразования. Анализ страничного поведедения процессов: свойство локальности, рабочее множество.}

\newpage

\subsection{Виртуальная память: управление памятью страницами по запросу – три схемы преобразования; реализация страничного преобразования в компьютерах на базе процессоров Intel (x86): стандартное преобразование и PAE в защищенном режиме – схемы, размеры таблиц и их количество на каждом этапе преобразования. Сегментно-страничное распределение памяти по запросам (сегментами, разделенными на страницы по запросу)}

\newpage

\subsection{Вирутуальная память: управление памятью страницами по запросам – три схемы. Алгоритмы вытеснения страниц: демонстрация особенностей на модели траектории страниц. Рабочее множество – определение, глобальное и локальное размещение. Флаги в дескрипторах страниц, предназначенные для реализации замещения страниц.}

\newpage

\subsection{Виртуальная память: распределение памяти страницами по запросам, свойство локальности, рабочее множество, анализ страничного поведения процессов. Схема страничного преобразования в процессорах Intel (x86) PAE – размеры таблиц и дескрипторов. Обоснование использования многоуровневого преобразования, кэш TLB – струкутра (процессор 486).}

\newpage

\subsection{Вирутуальная память: управление памью страницами по запросам – три схемы преобразования виртуального адреса к физическому. Алгоритмы вытеснения страниц: демонстрация особенностей на модели траектории траниц. Рабочее множество – определение, глобальное и локальное замещение. Флаги в дескрипторах страниц, предназанченные для реализации замещения страниц.}

\newpage

\subsection{Управление виртуальной памятью. Распределение памяти сегментами по запросам: схема преобразования виртуального адреса, способы организации таблиц сегментов, стратегии выбора разделов памяти для загрузки сегментов, алгоритмы и особенности замещения сегментов.}

\newpage

\subsection{Управление памятью сегментами по запросам в архитектуре x86. Организация таблиц сегментов в защищенном режиме. Формат дескриптора сегмента в таблицах дескрипторов сегментов (GDT и LDT) (заполнение полей дескрипторов GDT из лабораторной работы по защищенному режиму).}
