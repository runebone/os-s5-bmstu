\section{Проблемы распараллеливания. Тупики. Философы. Булочная. Лампорт.}

\subsection{Тупики: Обнаружение тупиков для повторно используемых ресурсов методом редукуции графа, способы представления графа, алгоритмы обнаружения тупиков. Пример анализа состояния системы метод редукции графа. Методы восстановления работоспособности системы.}

\newpage

\subsection{Тупики: классификация ресурсов и их особености. Четыре условия возникновения тупика. Методы исключения тупиков.}

\newpage

\subsection{Задача "Обедающие философы" – модели распределения ресурсов вычислительной системы. Множественные семафоры UNIX: системные вызовы, поддержка в системе, пример использования из лабораторной работы "производство-потребление".}

\newpage

\subsection{Процессы: взаимодействие параллельных процессов – монопольный доступ и взаимоисключение; программная реализация взаимоисключения – флаги, алгоритм Деккера, алгоритм Лампорта "Булочная".}

\newpage

\subsection{Процессы: взаимодействие параллельных процессов – монопольный доступ и взаимоисключение; алгорит Лампорта "Булочная"и "Логические часы" Лампорта.}

\newpage

\subsection{Процессы: бесконечное откладывание, зависание, тупиковая ситуация – анализ на примере задачи об обедающих философах и примеры аналогичных ситуаций в ОС. Множественные семафоры в Linux: системные вызовы и поддержка в ОС Linux; примеры из лабораторных работ.}

\newpage

\subsection{Процессы: бесконечное откладывание, зависание, тупиковая ситуация – анализ на примере задачи об обедающих философах и примеры аналогичныхх ситуаций в ОС. Множественные семафоры в Linux: системные вызовы и поддержка в системе; пример из лабораторной работы "производство-потребление".}
